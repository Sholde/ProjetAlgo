\documentclass{article}
\usepackage[utf8]{inputenc}

\title{ProjetAlgo}
\author{Nicolas, Taariq, Théo}
\date{April 2019}

\begin{document}

\maketitle

\section{Introduction}

Le but de ce projet est de créer un une application qui calcul la table de routage de chaque noeuds d'un réseaux de 100 noeuds( graphe de 100 noeuds ). 
\\Nous avons décidé de faire ce projet avec le language C.

\section{Structure utlilisé}

Nous avons décidé d' utilisé des pointeurs vers les structures au lieux de les passé en argument sans pointeurs.

\subsection{Graphe}

Cette structure contient 2 champs :
\begin{itemize}
    \item un tableau qui représente le graphe
    \item une structure qui permet de bien initialisé le graphe
\end{itemize}

\subsection{Tableau du graphe}

Ce tableau représente le graphe. Si il y a une arrête entre i et j alors le poids de l'arrête est noté , sinon il y a -1 dans list[i][j] et list[j][i].

\subsection{Insert}

Cette structure permet de bien initialisé le graphe et contient 2 champs :
\begin{itemize}
    \item un tableau qui compte le nombre d'arrête vers le même tier
    \item un tableau où est noté le nombre d'arrête qu'il faut avoir vers le même tier ( est utilisé uniquement pour le tier 2 et 3 )
\end{itemize}

\subsubsection{Compteur}

Ajoute 1 a la valeut au sommet i a chaque fois qu' on ajoute une arrête vers le même tier.

\subsubsection{Proba}

Ce tableau est initialisé au moment ou on initialise le graphe et permet de savoir combien d' arrêtes il peut y avoir vers un noeuds du même tier. Pour le tier 2 et 3 lorsque on ajoute une arrête entre le sommet i et j il faut aussi l'ajouter de j vers i et sauvegarder qu' il y a une arrête vers le même tier.

\subsection{Table de routage}

Cette structure contient 2 champs : 
\begin{itemize}
    \item un tableau qui représente la table de routage avec le poids
    \item un tableau qui représente les père 
\end{itemize}

\subsubsection{Poids}

Ce tableau représente le poids minimum pour aller d'un sommet vers un autre. Par exemple prenons i le sommet de départ et j le sommet d' arrivé, poids[i][j] indique le poids minimum pour aller de i à j.

\subsubsection{Père}

Ce tableau reprèsente le père du sommet d' arrivé en prenant compte du plus petit chemin.\\
Par exemple on prend 3 sommet i, j et k. Le chemin le plus court pour aller de i à j est : i \begin{math}\rightarrow\end{math} ... \begin{math}\rightarrow\end{math} k \begin{math}\rightarrow\end{math} j. Donc dans le tableau pere[i][j] il y aura k.

\section{Création du graphe}
à faire
\section{Vérification de la connexité}
à faire
\section{Création de la table de routage}
à faire
\section{Reconstitution du chemin}
à faire
\end{document}
